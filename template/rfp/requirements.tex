%
% حق نشر 1392-1402 دانش پژوهان ققنوس
% حقوق این اثر محفوظ است.
% 
% استفاده مجدد از متن و یا نتایج این اثر در هر شکل غیر قانونی است مگر اینکه متن حق
% نشر بالا در ابتدای تمامی مستندهای و یا برنامه‌های به دست آمده از این اثر
% بازنویسی شود. این کار باید برای تمامی مستندها، متنهای تبلیغاتی برنامه‌های
% کاربردی و سایر مواردی که از این اثر به دست می‌آید مندرج شده و در قسمت تقدیر از
% صاحب این اثر نام برده شود.
% 
% نام گروه دانش پژوهان ققنوس ممکن است در محصولات به در آمده شده از این اثر درج
% نشود که در این حالت با مطالبی که در بالا اورده شده در تضاد نیست. برای اطلاع
% بیشتر در مورد حق نشر آدرس زیر مراجعه کنید:
% 
% http://dpq.co.ir/licence
%

%  
% Step 2: Capture Stakeholder Requirements
% 
% Ask each of these key stakeholders, or groups of stakeholders, for their
% requirements from the new product or service. What do they want and expect from
% this project?
% 
% Tip 1:Remember, each person considers the project from his or her individual
% perspective. You must understand these different perspectives and gather the
% different requirements to build a complete picture of what the project should
% achieve.
% 
% Tip 2:When interviewing stakeholders, be clear about what the basic scope of the
% project is, and keep your discussions within this. Otherwise, end-users may be
% tempted to describe all sorts of functionality that your project was never
% designed to provide. If users have articulated these desires in detail, they may
% be disappointed when they are not included in the final specification.
% 
% You can use several methods to understand and capture these requirements. Here,
% we give you four techniques:
% 
% Technique 1: Using stakeholder interviews
% 
%     Talk with each stakeholder or end-user individually. This allows you to
%     understand each person's specific views and needs.
%     
% Technique 2: Using joint interviews or focus groups
% 
%     Conduct group workshops. This helps you understand how information flows
%     between different divisions or departments, and ensure that hand-overs will
%     be managed smoothly.
%     
%     Tip:When using these two methods, it's a good idea to keep asking "Why?" for
%     each requirement. This may help you eliminate unwanted or unnecessary
%     requirements, so you can develop a list of the most critical issues.
% 
% Technique 3: Using "use cases"
% 
%     This scenario-based technique lets you walk through the whole system or
%     process, step by step, as a user. It helps you understand how the system or
%     service would work. This is a very good technique for gathering functional
%     requirements, but you may need multiple "use cases" to understand the
%     functionality of the whole system.
%     
%     Tip: You might want to find existing use cases for similar types of systems
%     or services. You can use these as a starting point for developing your own
%     use case.
% 
% Technique 4: Building Prototypes
% 
%     Build a mock-up or model of the system or product to give users an idea of
%     what the final product will look like. Using this, users can address
%     feasibility issues, and they can help identify any inconsistencies and
%     problems.
% 
% You can use one or more of the above techniques to gather all of the
% requirements. For example, when you have a complete list of requirements after
% your interviews, you can then build a prototype of the system or product.
% 

% Step 3: Categorize Requirements
% 
% To make analysis easier, consider grouping the requirements into these four
% categories:
% 
%     Functional Requirements 
%     Operational Requirements
%     Technical Requirements
%     Transitional Requirements
\section{\glspl*{requirement}}

\input{requirements/functional}
\input{requirements/operational}
\input{requirements/technical}
\input{requirements/transitional}